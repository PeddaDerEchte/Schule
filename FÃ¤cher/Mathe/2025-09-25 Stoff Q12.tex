\documentclass[a4paper,12pt]{article}

\usepackage{hyperref}
\usepackage[utf8]{inputenc}
\usepackage[ngerman]{babel}
\usepackage[T1]{fontenc}
\usepackage{amsmath}
\usepackage{amsfonts}
\usepackage{amssymb}
\usepackage{lmodern}
\usepackage[onehalfspacing]{setspace}
\usepackage[left=3.5cm,right=2.5cm,top=2.5cm,bottom=2cm]{geometry}
\usepackage{graphicx}
\usepackage{ragged2e}
\usepackage{microtype}
\usepackage{cleveref}
\usepackage{parskip}
\usepackage{tabularx}
\usepackage{float}
\usepackage{siunitx}
\usepackage[all]{nowidow}
\usepackage{array}
\usepackage{caption}
\usepackage{csquotes}
\usepackage{upgreek}
\usepackage{mleftright}
\usepackage{biblatex}
\usepackage{enumerate}
\usepackage{esvect}


%Neue Befehle
\newcommand{\N}{\mathbb{N}}
\newcommand{\Z}{\mathbb{Z}}
\newcommand{\Q}{\mathbb{Q}}
\newcommand{\R}{\mathbb{R}}
\newcommand{\C}{\mathbb{C}}
\newcommand{\e}{\mathrm{e}}
%\newcommand{\i}{\mathrm{i}}
\newcommand{\qed}{\hfill $\blacksquare$}
\crefname{section}{Kapitel}{Kapitel}
\crefname{subsection}{Kapitel}{Kapitel}
\crefname{subsubsection}{Kapitel}{Kapitel}
\Crefname{section}{Kapitel}{Kapitel}
\Crefname{subsection}{Kapitel}{Kapitel}
\Crefname{subsubsection}{Kapitel}{Kapitel}
\crefname{figure}{Abbildung}{Abbildungen}


%Dokument
\begin{document}

\title{Script für das bayerische Abitur im Fach Mathematik (eA)}
\author{Nico Schneider}
\maketitle
\thispagestyle{empty}
\newpage

\tableofcontents
\thispagestyle{empty}

\newpage

\section{Stammfunktionen und weitere Ableitungsregeln}

\subsection{Funktionsscharen}

\subsubsection{Definition}
Hat die Funktion $f:x\mapsto f(x)$ eine von $x$ unabhängige Variable $t$, so existiert zu jedem Wert des \emph{Parameters} $t$ eine Funktion $f_t:x\mapsto f_t(x)$. Die Menge aller Funktionen $f_t:x\mapsto f_t(x)$ wird als \emph{Funktionenschar} bezeichnet.

\subsection{Stammfunktionen}

\subsubsection{Definition}

Sei $F:x\mapsto F(x)$ eine differenzierbare Funktion, so heißt die Funktion $F$ genau dann Stammfunktion von $f$, wenn für alle $x\in D_f$ gilt:
\[ F'(x) = f(x)\]


\subsubsection{Satz}
Sei $f:x\mapsto f(x)$ eine in $\R$ definierte Funktion und $C \in \R$, so gilt zum Aufleiten der Funktion $f$:
\begin{enumerate}
\item $f(x) = x^n \ (n \in \N )\  \Rightarrow \  F(x)=\dfrac{1}{n+1}\cdot x^{n+1} +C$
\item $f(x)=\cos(x)\ \Rightarrow \ F(x)=\sin(x) + C$
\item $f(x)=\sin(x) \ \Rightarrow \ F(x)=-\cos(x) + C$
\end{enumerate}

\subsection{Ableitung von Sinus- und Kosinusfunktion}
\subsubsection{Satz}
Sei $f:x\mapsto \sin (x), x\in\R$, so gilt:
\[ f'(x)=\cos(x), \ f''(x) = -\sin(x), \ f'''(x)=-\cos(x) \ \text{und} \ f^{4}(x)=\sin(x)\]

\subsection{Produktregel}
\subsubsection{Satz (Produktregel)}
Sind die Funktionen $u(x)$ und $v(x)$ mit $x\in D$ differenzierbar, so gilt für die Funktion $f(x)=u(x)\cdot v(x)$:
\[ f'(x)=u'(x)\cdot v(x) + u(x)\cdot v'(x) \]

\subsubsection{Beweis}
Gegeben sei die Funktion $f$ mit $f(x) = u(x) \cdot v(x)$. Die Ableitung von $f$ an einer Stelle $x$ ist dann durch den Grenzwert des Differenzenquotienten
\[ \lim \limits_{h \rightarrow 0} \dfrac{u(x+h)\cdot v(x+h)-u(x)\cdot v(x)}{h} \]
gegeben. Durch Addition und Subtraktion des Terms $\dfrac{u(x)\cdot v(x+h)}{h}$ erhalten wir
\[ \lim \limits_{h \rightarrow 0} \dfrac{u(x+h)\cdot v(x+h)- u(x)\cdot v(x+h) + u(x)\cdot v(x+h) - u(x)\cdot v(x)}{h} \text{.} \]
Durch geschicktes Umformen vereinfachen wir und kommen auf
\[ f'(x)=u'(x)\cdot v(x) + u(x) \cdot v'(x) \text{.}\]
\qed

\subsection{Verkettung von Funktionen und Kettenregel}
\subsubsection{Definition}
Für zwei Funktionen $u$ und $v$ gibt es zwei mögliche Verkettungen:\\
Die \emph{Verkettung u nach v}: $u \circ v:x \mapsto u(v(x))$.\\
Die \emph{Verkettung v nach u}: $v \circ u:x \mapsto v(u(x))$.\\
Das Verketten zweier Funktionen ist allgemein \emph{nicht} kommutativ! 

\subsubsection{Satz (Kettenregel)}
Sei $f =u \circ v$ und weiterhin $u$ und $v$ differenzierbar, so ist auch $f(x)= u(v(x))$ differenzierbar und es gilt
\[ f'(x)=u'(v(x)) \cdot v'(x) \text{.}\]

\newpage
\section{Natürliche Exponentialfunktion}
\subsection{Die natürliche Exponentialfunktion und ihre Ableitung}

\subsubsection{Definition}
Wir definieiren die Euler'sche Zahl $\e$ wiefolgt:
\[ \e = \lim \limits_{n \rightarrow \infty}  \mleft( 1+\frac{1}{n} \mright) ^n = \sum\limits_{i=0}^{\infty} \dfrac{1}{i!}\]

\subsubsection{Satz}
Die Euler'sche Zahl $\e$ ist irrational: $\e\notin \Q$. 

\subsubsection{Definition}
Sei $f:x\mapsto \e^x$, so heißt $f$ \emph{natürliche Exponentialfunktion}.

\subsection{Ableitungsregeln zur natürlichen Exponentialfunktion}

\subsubsection{Satz}
Sei $f:x\mapsto \e^x$ mit $x\in\R$, so gilt $f':x\mapsto \e^x$.

\subsubsection{Bemerkung}
\begin{enumerate}[(1)]
\item Alle Funktionen $F_C : x \mapsto \e^x+C$ mit $C\in\R$ sind also Stammfunktionen von $f$.
\item Die Ableitung einer Funktion $f_a :x \mapsto a \cdot \e^x$ ist nach der Produktregel also $f_a':x\mapsto a \cdot \e^x = f_a(x)$
\end{enumerate}

\subsubsection{Satz}
Sei $v(x)$ eine differenzierbare Funktion und weiterhin $f(x)$ und $g(x)$ die verknüpften Funktionen $f(x)=\e^{v(x)} \ \text{bzw.} \ g(x)=\e^x \cdot v(x)$. So gilt:
\begin{enumerate}[(i)]
\item $f'(x)=\e^{v(x)}\cdot v(x)$ (Kettenregel)
\item $g'(x)=\e^x\cdot(x)+\e^x\cdot v'(x)=\e^x \cdot \mleft( v(x)+v'(x) \mright)$
\end{enumerate}

\subsubsection{Bemerkung}
Im Falle von (i) überträgt sich das Monotonieverhalten der Funktion $v(x)$ auf die Funktion $f(x)$. 

\subsection{Grenzwerte von Verknüpfungen mit der natürlichen Exponentialfunktion}
\subsubsection{Satz}
Es gelten folgende Grenzwerte von Verknüpfungen mit der natürlichen Exponentialfunktion ($n \in \N$):
\begin{enumerate}[(i)]
\item $\lim \limits_{x \rightarrow - \infty} (x^n\cdot e^x) = 0$
\item $\lim \limits_{x \rightarrow + \infty} \dfrac{x^n}{e^x} = \lim \limits_{x \rightarrow - \infty} (x^n\cdot e^{-x}) = 0$
\item $\lim \limits_{x \rightarrow + \infty} (e^x-x^n)=+ \infty$
\end{enumerate}

\subsubsection{Bemerkung}
\begin{enumerate}[(1)]
\item Anschaulich bedeutet das also, dass $\e^x$ für $x \rightarrow \infty$ \glqq schneller\grqq \ wächst als jede Potenz $x^n$.
\item Da das Grenzwertverhalten einer ganzrationalen Funktion $p: x \mapsto a_nx^n + a_{n-1}x^{n-1}+...+a_1x+a_0$ nur von Summanden mit der höchsten Potenz $a_nx^n$ bestimmt wird, gelten die Grenzwerte anstelle von einer Potenz $x^n$ auch für Polynome $p(x)$.
\end{enumerate}

\subsection{Natürlicher Logarithmus und Exponentialgleichungen}
\subsubsection{Definition}
Die Lösung der \emph{Exponentialgleichung} $\e^x = b$  mit $b \in \R ^+$ und $x \in \R$ bezeichnet man als \emph{natürlichen Logarithmus von b}. Man schreibt:
\[ x = \ln (b) \Leftrightarrow  \e^x = b \]

\subsubsection{Bemerkung}
\begin{enumerate}[(1)]
\item Aus der Definition des natürlichen Logarithmus folgt $\e^{\ln(x)}=x$ für $x>0$ und $\ln(\e^x)=x$.
\item Weiterhin gilt $\ln(1)=0$ und $\ln(e)=1$.
\item Für das Rechnen mit dem natürlichen Logarithmus gilt $\ln(b^x)=x\cdot \ln(b)$.
\end{enumerate}

\subsection{Modellieren von Wachstums- und Abklingvorgängen}
\subsubsection{Definition}
Wir nennen die Funktion $f:t\mapsto b \cdot a^t (a\in\R^+ \textbackslash \{ 1 \} )$ mit Wachstumsfaktor $a$ und Anfangsbestand $b=f(0)$ \emph{Wachstumsfunktion}. Eine alternative Darstellung ist:
\[ f:t\mapsto b \cdot \e^{\ln(a)\cdot t}  \] 

\subsubsection{Satz}
Aus der Wachstumsfunktion $f:t\mapsto \e^{k\cdot t}$ ergibt sich unmittelbar die Halbwerts- bzw. Verdopplungszeit.
\begin{enumerate}[(i)]
\item $k>0: T_V=\frac{\ln(2)}{k}$
\item $k<0: T_H=-\frac{\ln(2)}{k}$
\end{enumerate}

\newpage
\section{Zufallsgrößen und Binomialverteilung}
\subsection{Zufallsgrößen und deren Wahrscheinlichkeitsverteilung}
\subsubsection{Definition}
Eine Funktion $X$, die jedem Ergebnis $\omega \in \Omega$ eines Zufallsexperiments eine reelle Zahl $X(\omega)$ zuordnet, heißt \emph{Zufallsgröße} oder \emph{Zufallsvariable} auf $\Omega$. Kurz: $X:\omega \mapsto X(\omega)$ mit $\omega \in \Omega$ und $X(\omega) \in \R$.

\subsubsection{Definition}
Die Funktion $W$, die jedem Wert $x_i$ einer Zufallsgröße $X$ die Wahrscheinlichkeit $P(X=x_i)$ zuordnet, heißt \emph{Wahrscheinlichkeitsfunktion der Zufallsgröße $X$} oder \emph{Wahrscheinlichkeitsverteilung der Zufallsgröße $X$} oder knapp Verteilung von $X$. Kurz: $W:x_i \mapsto P(X=x_i)$.

\subsubsection{Definition}
Die Funktion $F$, die bei gegebener Zufallsgröße $X$ jeder reellen Zahl $x$ die Wahrscheinlichkeit $P(X\leq x)$ zuordnet, heißt \emph{kumulative Verteilungsfunktion der Zufallsgröße $X$}. Kurz: $F:x\mapsto P(X\leq x)$ mit $x \in \R$ und $P(X\leq x) \in [0; 1]$.

\subsection{Erwartungswert und Varianz einer Zufallsgröße}
\subsubsection{Definition}
\[ E(x) = \mu = \sum\limits_{i=1}^{n} x_i \cdot P(X=x_i) \]
\underline{In Worten:} Ist $X$ eine Zufallsgröße, deren mögliche Werte $x_1, x_2, ..., x_n$ sind, so heißt die reelle Zahl $E(X)=x_1 \cdot P(X=x_1) + x_2 \cdot P(X=x_2) + ... + x_n\cdot P(X=x_n)$ \emph{Erwartungswert der Zufallsgröße X}. Der Erwartungswert ist immer der mittlere Wert der Zufallsgröße pro Versuch auf lange Sicht. 

\subsubsection{Bemerkung}
Ein Spiel heißt \emph{fair}, wenn der Erwartungswert des Gewinns für jeden Spieler 0 ist.


\subsubsection{Definition}
Ist $X$ eine Zufallsgröße, deren mögliche Werte $x_1, x_2,..., x_n$ sind und die den Erwartungswert $E(X)=\mu$ hat, so heißt die reelle Zahl $Var(X)=(x_1-\mu)^2\cdot P(X=x_1)+(x_2-\mu)^2 \cdot P(X=x_2)+ ... + (x_n-\mu )^2 \cdot P(X=x_n)$ \emph{Varianz der Zufallsgröße $X$}.

\subsubsection{Definition}
Die reelle Zahl $\sqrt{Var(X)}$ heißt \emph{Standardabweichung der Zufallsgröße $X$}.

\subsubsection{Bemerkung}
Statt $\sqrt{Var(X)}$ schreibt man auch $\sigma$ und folglich gilt für die Varianz $Var(X)=\sigma^2$.

\subsection{Ziehen aus einer Urne mit Beachtung der Reihenfolge}
\subsubsection{Satz}
Unter der Beachtung der Reihenfolge unterscheidet man grundlegend zwei Zufallsexperimente. Für das Ziehen aus einer Urne mit $n$ unterscheidbaren Kugeln und $k$-maligem Ziehen gilt...
\begin{enumerate}[(i)]
\item ...mit Zurücklegen: Es sind $n^k$ Ergebnisse möglich. 
\item ...ohne Zurücklegen: Es sind $n\cdot (n-1)\cdot ... \cdot (n-k+1)$ verschiedene Ergebnisse möglich. 
\end{enumerate}

\subsubsection{Bemerkung}
Jede Anordnung einer Reihenfolge wird als \emph{Permutation} bezeichnet.

\subsection{Ziehen aus einer Urne ohne Beachtung der Reihenfolge}
\subsubsection{Satz}
Zieht man ohne Zurücklegen und ohne Beachtung der Reihenfolge aus $n$ unterschiedlichen Kugeln $k$-mal, so gibt es \[ \dfrac{n\cdot (n-1)\cdot (n-2) \cdot ... \cdot (n-k+1)}{k!} \] mögliche Ergebnisse.

\subsubsection{Definition (Binomialkoeffizient)}
Für $k, n \in \N$ und $k\leq n$ heißt $ \dbinom{n}{k} = \dfrac{n!}{k! \cdot (n-k)!}$ \emph{Binomialkoeffizient}.

\subsubsection{Bemerkung}
\begin{enumerate}[(1)]
\item Es ist bekannt, dass $0!=1$ gilt.
\item Wir erkennen die Symmetrie $\binom{n}{n-k}=\binom{n}{k}$ des Binomialkoeffizienten. Der Beweis kann durch Umformen nachvollzogen werden und ist dem Leser überlassen.
\end{enumerate}

\subsubsection{Satz}
Aus einer Urne mit $N$ Kugeln, von denen $S$ schwarz sind, werden $n$ Kugeln ohne zurücklegen und ohne Beachtung der Reihenfolge, z.B. mit einem Griff, gezogen. Die Zufallsgröße $X$ gibt die Anzahl $s$ der gezogenen schwarzen Kugeln an.  Dann gilt: 
\[ 
P(X=s)=\dfrac{\dbinom{S}{s}\cdot \dbinom{N-S}{n-s}}{\dbinom{N}{n}}
\]


\subsection{Bernoulli-Experimente mit Bernoulli-Ketten}
\subsubsection{Definition}
Ein Zufallsexperiment mit genau zwei Ergebnissen heißt \emph{Bernoulli-Experiment}. Die Wahrscheinlichkeit für einen Treffer wird mit $p$, die für eine Niete mit $q$ bezeichnet, wobei $q=1-p$ ist. Ein Zufallsexperiment, das aus $n$ unabhängigen Durchführungen desselben Bernoulli-Experiments besteht, heißt \emph{Bernoulli-Kette} der Länge $n$ mit dem Parameter $p$. 

\subsection{Binomialverteilung}
\subsubsection{Satz (Formel von Bernoulli)}
Gegeben ist eine Bernoulli-Kette der Länge $n$, $n \in \N \textbackslash \{ 0 \}$ mit Trefferwahrscheinlichkeit $p$. Die Zufallsgröße $X$ gibt die Anzahl der Treffer an. Dann beträgt die \emph{Wahrscheinlichkeit für genau $k$ Treffer} mit $k \in \{ 0; 1; ...; n \}$: 
\[ P(X=k)=\dbinom{n}{k}\cdot p^k \cdot (1-p)^{n-k} \]

\subsubsection{Bemerkung}
Die Schreibweisen
\begin{enumerate}[(a)]
\item $B(n;p)$,
\item $ P(X=x)$ und
\item $P_p^n(X=x)$
\end{enumerate}
sind äquivalent.


\subsubsection{Definition}
Eine Zufallsgröße $X$ heißt \emph{binomialverteilt} nach $B(n;p)$ oder $B(n;p)$-verteilt, wenn gilt
\begin{enumerate}[(i)]
\item $X$ kann die Werte $0;1;2;...;n$ mit $n \in \N \textbackslash \{ 0 \}$ annehmen und
\item $P(X=k)=\dbinom{n}{k}\cdot p^k \cdot (1-p)^{n-k}$ mit $0\leq p \leq 1$.
\end{enumerate}

\subsubsection{Vorgehen beim Modellieren}
\begin{itemize}
\item Man prüft, ob das Zufallsexperiment als Bernoulli-Kette der Länge $n$ mit der Wahrscheinlichkeit $p$ für Treffer angesehen werden kann.
\item Ist dies der Fall, führt man eine $B(n;p)$-verteilte Zufallsgröße $X$ ein.
\item Man bestimmt die gesuchte Wahrscheinlichkeit mithilfe der Binomialverteilung.
\end{itemize}

\subsection{Erwartungswert und Varianz der Binomialverteilung}
\subsubsection{Satz}
Für die nach Kenngrößen einer nach $B(n;p)$-verteilten Zufallsgröße $X$ gilt
\begin{enumerate}[(i)]
\item für den Erwartungswert $\mu = n\cdot p$,
\item für die Varianz $\sigma^2 = n\cdot p \cdot q$ mit $q=1-p$ und 
\item für die Standardabweichung $\sigma = \sqrt{n \cdot p\cdot q}$.
\end{enumerate}

\subsection{Axiomatische Definition von Wahrscheinlichkeit}
\subsubsection{Definition (Axiome von Kolmogorow)}
Eine Funktion $P:A\mapsto P(A)$ mit $A \subset \Q$ und $P(A) \in \R $ heißt \emph{Wahrscheinlichkeitsverteilung} und $P(A)$ \emph{Wahrscheinlichkeit} von $A$, wenn $P$ folgende Bedingungen erfüllt sind:
\begin{enumerate}[(i)]
\item $P(A) \geq 0$
\item $P(\Omega)=1$
\item $A \cap B = \{ \} \Rightarrow P(A \cup B)= P(A) + P(B)$
\end{enumerate}

\newpage
\section{Beurteilende Statistik}
\subsection{Einseitiger Hypotesentest}
\subsubsection{Grundwissen}
Zu einem Sachverhalt werden zwei sich ausschließende Hypothesen betrachtet: die \emph{Nullhypothese} $H_0$ und die \emph{Gegenhypothese} $H_1$. Die Anzahl der Treffer einer \emph{Stichprobe} mit festgelegter Länge bildet die \emph{Testgröße}. Der Wertebereich der Testgröße wird in den \emph{kritischen Bereich} $K$ (\emph{Ablehnungsbereich}) und den \emph{nichtkritischen Bereich} $\overline{K}$ zerlegt. Liegt der durch die Stichprobe gewonnene Wert der Testgröße in $K$, dann wird $H_0$ verworfen, ansonsten wird $H_0$ nicht verworfen (\emph{Entscheidungsregel}).

\subsubsection{Bemerkung}
\begin{enumerate}[(1)]
\item Es ist möglich, dass die Nullhypothese verworfen wird, obwohl sie in Wirklichkeit wahr ist.
\item Ist $K$ links von $\overline{K}$, so spricht man von einem linksseitigen Hypothesentest. Andersherum, also wenn $K$ rechts von $\overline{K}$ liegt, spricht man von einem rechtsseitigen Hypothesentest.
\end{enumerate}

\subsubsection{Definition}
Einen Hypothesentest mit einem kritischen Bereich $K$, der aus einem einzigen Intervall besteht, bezeichnet man als \emph{einseitigen Hypothesentest}.
\par
\begin{center}

\begin{tabular}{ |c|c| }
\hline
Linksseitiger Hypothesentest & Rechtsseitiger Hypothesentest \\
\hline
$H_0:p=p_0$ (oder $p\geq p_0$) & $H_0:p=p_0$ (oder $p\leq p_0$) \\
$H_1: p<p_0$ & $H_1: p>p_0$ \\
\hline
\end{tabular}
\end{center}
\subsubsection{Fehler erster Art}
Die Nullhypothese $H_0$ wird abgelehnt ($Z \in K$), obwohl sie eigentlich wahr ist. Fehler erster Art werden mit $\alpha'$ bezeichnet. Für die Berechnung genügt es bei einseitigen Hypothesentests, mit der Wahrscheinlichkeit $p=p_0$ zu rechnen, da alle Werte bei einem linksseitigen Hypothesentest für $p>p_0$ (rechtsseitig: $p<p_0$) eine kleinere Wahrscheinlichkeit haben.
\subsubsection{Fehler zweiter Art}
Die Nullhypothese $H_0$ wird nicht abgelehnt ($Z \in \overline{K}$), obwohl eigentlich die Gegenhypothese $H_1$ zutrifft. Fehler zweiter Art werden mit $\beta'$ bezeichnet. Man kann $\beta'$ nur berechnen, wenn für die Testgröße $Z$ eine bestimmte Wahrscheinlichkeitsverteilung (und damit auch $p$) angenommen wird. 

\subsubsection{Einflussnahmen auf $\alpha'$ und $\beta'$}
Vergrößert man den \emph{Ablehnungsbereich}, wird $\beta'$ zwar kleiner, aber $\alpha'$ damit zwangsläufig größer und umgekehrt. 
Durch die Vergrößerung des \emph{Stichprobenumfangs} kann man Fehler erster Art als auch Fehler zweiter Art verkleinern. 

\subsection{Konstruktion eines einseitigen Signifikanztestes}
\subsubsection{Definition}
Ein Hypothesentest mit einem zuvor festgelegten \emph{Signifikanztest} $\alpha$ und einem Ablehnungsbereich $K$, der aus einem einzigen Intervall besteht, bezeichnet man als \emph{einseitigen Signifikanztest}.

\subsubsection{Bemerkung}
Meist wird für $\alpha $ der Wert $1 \%, 2 \%$ oder $5 \%$ angenommen.


\subsubsection{Konstruktion eines einseitigen Signifikanztests}
\begin{enumerate}[(1)]
\item Festlegen der Testgröße $Z$ und des Stichprobenumfangs $n$.
\item Mathematische Formulierung der Nullhypothese $H_0$ und der Gegenhypothese $H_1$.
\item Festlegen des Signifikanzniveaus $\alpha$ gemäß der in der Anwendungssituation maximal tolerierten Wahrscheinlichkeit für den Fehler erster Art.
\item Bestimmen der Entscheidungsregel, d.h. Konstruktion des Ablehnungsbereichs $K$.
\end{enumerate}
\begin{tabularx}{\textwidth}{ |X|X| }
\hline
\emph{Linksseitiger Test:} & \emph{Rechtsseitiger Test} \\
Ablehnungsbereich $K=\{ 0; 1; ...; g \}$ & Ablehnungsbereich $K=\{ g; g+1; ...; n \}$ \\
wobei $g$ die größte natürliche Zahl ist mit & wobei  $g$ die kleinste natürliche Zahl ist mit \\
$\alpha'=P_{p_0}^n(Z\leq g)\leq \alpha$ & $\alpha'=P_{p_0}^n(Z\geq g)\leq \alpha$ \\
\hline
\end{tabularx}

\subsubsection{Bemerkung}
Beim rechtsseitigen Signifikanztest muss man beim Ermitteln von $g$ auf Folgendes achten: 
\[ \alpha' \leq \alpha \Leftrightarrow P_{p_0}^n(Z\geq g) \leq \alpha \Leftrightarrow 1-P_{p_0}^n(Z\leq g - 1) \leq \alpha \Leftrightarrow P_{p_0}^n (Z\leq g - 1) \geq 1 - \alpha  \]
Für Bestimmung von $g$ muss auf $g-1$ noch $1$ addiert werden. 


\newpage
\section{Quotientenregel und Funktionsuntersuchungen}
\subsection{Die Quotientenregel bei gebrochen-rationalen Funktionen}
\subsubsection{Satz (Quotientenregel)}
Sind die Funktionen $u$ und $v$ differenzierbar, dann ist $f:x\mapsto \dfrac{u(x)}{v(x)}$ für alle $x$ mit $v(x) \neq 0$ differenzierbar und es gilt $f'(x)=\dfrac{u'(x)\cdot v(x)-u(x)\cdot v'(x)}{[ v(x) ]^2}$.

\subsubsection{Satz}
Sei $f:x\mapsto x^k, \ (k\in\Z^-)$, so ist $f(x)$ auf $\R\textbackslash \{ 0 \}$ differenzierbar mit
\[ f'(x)= k\cdot x^{k-1} \text{.}\]

\subsubsection{Aspekte bei der Untersuchung einer Funktion $f$ und ihren Graphen $G_f$}
Beim Untersuchen einer Funktion $f:x\mapsto f(x)$ mit Definitionsmenge $D_f \subseteq \R$ gilt es, folgende Untersuchungen zu tätigen.
\begin{enumerate}[(1)]
\item Definitionsmenge $D_f$ von $f$
\item Symmetrieverhalten von $G_f$ (Ansatz: $f(-x)$ einsetzen)
\item Nullstellen von $f$ ($0=f(x)$ auflösen)
\item Schnittpunkte von $G_f$ mit der y-Achse (Untersuchung nach $f(0)$)
\item Verhalten von $f$ an den Rändern von $D_f$ (Grenzwerte betrachten)
\item Asymptoten von $G_f$ (beachte vertikale und horizontale, z.B. über Grenzwerte)
\item Extrempunkte von $G_f$ (Prüfen nach $f'(x)=0$ und $f''(x)\neq 0$)
\item Monotonieverhalten von $f$ (1. Ableitung)
\item Wendepunkte von $G_f$ ($f''(x)=0$ und $f'''(x)\neq 0$)
\item Krümmungsverhalten von $G_f$ (Prüfen ob $f''(x)$ größer oder kleiner als 0 ist)
\end{enumerate}


\subsubsection{Satz (Regel von L'Hôpital)}
Gegeben sei ein offenes Intervall $I$ mit einer Stelle $a\in I$. Seien $u$ und $v$ nun zwei Funktionen, die auf $I \textbackslash \{ a \}$ differenzierbar seien und außerdem gelte für alle $x \in I \textbackslash \{ a \}$ $v'(x)\neq 0$. Existiert nun der Grenzwert $\lim\limits_{x\rightarrow a} \frac{u'(x)}{v'(x)}$, so gilt:

\[ \mleft[ \mleft( \lim\limits_{x \rightarrow a}u(x) = \lim\limits_{x \rightarrow a}v(x) \mright) \lor \mleft( \lim\limits_{x \rightarrow a}u(x) = \lim\limits_{x \rightarrow a}v(x) = \pm \infty \mright) \mright] \Rightarrow \lim\limits_{x \rightarrow a}\dfrac{u(x)}{v(x)} = \lim\limits_{x \rightarrow a}\dfrac{u'(x)}{v'(x)}\]

$\lor$ heißt hierbei \glqq oder\grqq im mathematischen Sinne. 

\newpage
\section{Wurzelfunktion}
\subsection{Die Umkehrfunktion}
\subsubsection{Definition}
Sei $f:x\mapsto f(x)$ mit Definitionsmenge $D_f$ und Wertemenge $W_f$. $f(x)$ heißt genau dann \emph{umkehrbar}, wenn für alle $y\in W_f$ genau ein $x\in D_f$ mit $y=f(x)$ existiert. Diese umgekehrte Zuordnung nennt man \emph{Umkehrfunktion} von $f$ oder kurz $f^{-1}$. 

\subsubsection{Satz (Umkehrbarkeitskriterium}
Gegeben sei eine Funktion $f:x \mapsto f(x)$ im Intervall $I$, so ist $f(x)$ stets umkehrbar, wenn für alle $x\in I$ die Bedingung $f'(x)>0$ oder $f'(x)<0$ erfüllt ist.

\subsection{Potenzfunktionen mit rationalem Exponenten und ihre Ableitung}
\subsubsection{Definition}
Sei $f:x\mapsto a \cdot x^{\frac{p}{q}}=a\cdot \sqrt[q]{x^p}$ mit $a\in \R$, $p\in\Z$ und $q\in \N \textbackslash \{ 0 \}$, so heißt $f$ Potenzfunktion mit rationalem Exponenten. 

\subsubsection{Satz (Ableitungsregel für Potenzfunktionen}
\begin{enumerate}[(i)]
\item $f:x\mapsto x^r \Rightarrow f':x\mapsto r\cdot x^{r-1}$ ($r \in \Q$)
\item $f:x\mapsto x^r \Rightarrow F:x\mapsto \dfrac{1}{r+1}x^{r+1}$ ($r \in \Q \textbackslash \{ -1 \}$)
\end{enumerate}

\newpage
\section{Natürliche Logarithmusfunktion}
\subsection{Die natürliche Logarithmusfunktion und ihre Ableitung}
\subsubsection{Definition}
Sei $f:x\mapsto \ln(x)$ mit $D=\R^+$, so heißt $f$ natürliche Logarithmusfunktion.

\subsubsection{Satz}
Sei $f:x\mapsto \ln(x)$ mit $D_f=\R^+$, so gilt $f':x\mapsto \dfrac{1}{x}$ mit $D_{f'}=\R$.
Weiterhin gilt: $f:x\mapsto \dfrac{1}{x} \Rightarrow F:x\mapsto \ln(|x|)+C$ ($D_f=D_F=\R \textbackslash \{ 0 \}, C\in/R$

\subsubsection{Satz (Rechenregeln für den natürlichen Logarithmus)}
Sei $b,c\in\R^+$ und $r\in\R$, so gilt
\begin{enumerate}[(i)]
\item $\ln(b\cdot c) = \ln(b) + \ln(c)$,
\item $\ln(\frac{b}{c})=\ln(b)-\ln(c)$ und
\item $\ln(b^r)=r\cdot \ln(b)$.
\end{enumerate}

\subsection{Ableitung und Grenzwerte von Verknüpfungen mit der ln-Funktion}
\subsubsection{Satz}
Sei $v$ differenzierbar und $v'(x)>0$, so gilt
\[ f:x \mapsto \ln \mleft( v(x) \mright) \Rightarrow f':x\mapsto \dfrac{v'(x)}{v(x)} \text{.}\]


\subsubsection{Satz}
\[ \forall r\in \Q^+: \mleft[ \lim\limits_{x\rightarrow +\infty} \dfrac{\ln(x)}{x^r}=0 \land \lim\limits_{x\rightarrow 0}\mleft(x^r\cdot \ln(x) \mright)=0 \mright] \]
$\forall$ heißt hierbei: \glqq Für alle\grqq.

\section{Grundlagen der Koordinatengeometrie im Raum}
\subsection{Vektoren}
\subsubsection{Definition}
Die Menge aller gleichlanger, zueinander paralleler und gleichgerichteter Pfeile nennt man Vektor. Jeder Pfeil ist ein Repräsentant dieses Vektors. Man bezeichnet sie mit $\vv{v}$.
\par
Im zweidimensionalen Koordinatensystem hat jeder Vektor $\vv{v}$ zwei Koordinaten $\vv{v}=\mleft( \begin{array}{c}
x_1 \\
x_2
\end{array} \mright)$. Im Dreidimensionalen schreibt man $\vv{v}=\mleft( \begin{array}{c}
x_1 \\
x_2 \\
x_3
\end{array} \mright)$.

\subsubsection{Satz}
Zwischen zwei Punkten $A$ und $B$ konstruiert man den Vektor $\vv{AB}$ von $A$ nach $B$ wiefolgt:
$\vv{AB}= \mleft( \begin{array}{c}
b_1-a_1 \\
b_2 - a_2 \\
b_3 - a_3
\end{array} \mright)$

\subsubsection{Definition}
Der Pfeil vom Koordinatenursprung $O$ zum Punkt $B$ heißt Repräsentant des Ortsvektors $B$. $\vv{OB}=\mleft(  \begin{array}{c}
b_1 \\
b_2
\end{array}  \mright)$ bzw. $\vv{OB}=\mleft(  \begin{array}{c}
b_1 \\
b_2 \\
b_3
\end{array}  \mright)$

\subsubsection{Definition}
Die Länge eines Repräsentanten eines Vektors $\vv{v}$ nennt man Betrag des Vektors. 

\subsubsection{Satz}
Der Betrag eines Vektors $\vv{v}$ ist $|\vv{v}|= \sqrt{v_1^2+v_2^2}$ im Zweidimensionalen und $|\vv{v}|= \sqrt{v_1^2+v_2^2+v_3^2}$ im Dreidimensionalen.

\subsection{Addition und Subtraktion von Vektoren}
\subsubsection{Definition}
Zwei Vektoren $\vv{a}$ und $\vv{b}$ werden wiefolgt addiert: \par
$\vv{a} + \vv{b} = \mleft( \begin{array}{c} a_1 \\ a_2  \end{array} \mright) + \mleft( \begin{array}{c} b_1 \\ b_2  \end{array} \mright) = \mleft( \begin{array}{c} a_1+b_1 \\ a_2+b_2 \end{array} \mright) $ bzw. \par  $\vv{a} + \vv{b} = \mleft( \begin{array}{c} a_1 \\ a_2 \\a_3  \end{array} \mright) + \mleft( \begin{array}{c} b_1 \\ b_2 \\ b_3 \end{array} \mright) = \mleft( \begin{array}{c} a_1+b_1 \\ a_2+b_2 \\ a_3+b_3 \end{array} \mright) $. \par

Die Subtraktion $\vv{a}-\vv{b}$ kann als Addition mit dem Gegenvektor $-\vv{b}$ aufgefasst werden. So gilt: $\vv{a}-\vv{b} = \vv{a}+ -\vv{b}$.

\subsection{skalare Multiplikation eines Vektors}
\subsubsection{Definition}
Sei $\lambda\in\R$ eine beliebige reelle Zahl (\emph{Skalar}). Dann ist der Vektor $\lambda \cdot \vv{v}$ genau $|\lambda|$-mal so lang wie der Vektor $\vv{v}$ und parallel zu $\vv{v}$. Weiterhin wird definiert: $0 \cdot \vv{v}=\vv{0}$ und $\lambda \cdot \vv{0} = \vv{0}$. \par
Für einen Vektor $\vv{v}=\mleft( \begin{array}{c} v_1 \\ v_2 \\ v_3  \end{array} \mright)$ definiert man nun die skalare Multiplikation wiefolgt. 
\[ \lambda \cdot \vv{v} = \lambda \cdot \mleft( \begin{array}{c} v_1 \\ v_2 \\ v_3  \end{array} \mright) = \mleft( \begin{array}{c} \lambda \cdot v_1 \\ \lambda \cdot v_2 \\ \lambda \cdot v_3  \end{array} \mright) \]
Analog gilt für einen zweidimensionalen Vektor: 
\[ \lambda \cdot \vv{v} = \lambda \cdot \mleft( \begin{array}{c} v_1 \\ v_2  \end{array} \mright) = \mleft( \begin{array}{c} \lambda \cdot v_1 \\ \lambda \cdot v_2 \end{array} \mright) \]

\subsubsection{Definition}
Sei $\vv{v}$ ein beliebiger Vektor, so heißt $(-1) \cdot \vv{v} = -\vv{v}$ \emph{Gegenvektor} von $\vv{v}$. 

\subsubsection{Definition}
Sei $\vv{v}$ ein beliebiger vom Nullvektor verschiedener Vektor, so heißt der Vektor $\vv{v_0}= \dfrac{1}{|\vv{v}|}\cdot \vv{v}$ \emph{Einheitsvektor} von $\vv{v}$. Er hat stets die Länge $|\vv{v_0}|=1$.

\subsection{Skalarprodukt}
\subsubsection{Definition}
Für die Vektoren $\vv{a},\ \vv{b} \neq \vv{0}$ nennen wir die reelle Zahl $\vv{a} \circ \vv{b}=|\vv{a}| \cdot |\vv{b}| \cdot \cos (\varphi)$ \emph{Skalarprodukt} von $\vv{a}$ und $\vv{b}$. \par
Im $\R^2$ Vektorraum gilt:
\[ \vv{a} \circ \vv{b}=|\vv{a}| \cdot |\vv{b}| \cdot \cos (\varphi) = a_1\cdot b_1 + a_2 \cdot b_2 \]
Im dreidimensionalen Raum gilt analog dazu:
\[ \vv{a} \circ \vv{b}=|\vv{a}| \cdot |\vv{b}| \cdot \cos (\varphi) = a_1\cdot b_1 + a_2 \cdot b_2 + a_3 \cdot b_3 \]

\subsubsection{Satz}
Die Vektoren $\vv{a}, \ \vv{b} \neq \vv{0}$ sind genau dann orthogonal zueinander, wenn gilt $\vv{a} \circ \vv{b} = 0$.

\subsection{Das Kreuzprodukt (Vektorprodukt)}
\subsubsection{Definition}
Für die Vektoren $\vv{a}=\mleft( \begin{array}{c} a_1 \\ a_2 \\ a_3 \end{array}  \mright)$ und $\vv{b}=\mleft( \begin{array}{c} b_1 \\ b_2 \\ b_3 \end{array}  \mright)$ definieren wir das Kreuzprodukt als
\[ \vv{a} \times \vv{b} = \mleft( \begin{array}{c} a_2b_3-a_3b_2 \\ a_3b_1-a_1b_3 \\ a_1b_2-a_2b_1 \end{array}  \mright) /text{.}  \]

\subsubsection{Satz}
$(\vv{a} \times \vv{b}) \neq \vv{0} \Rightarrow (\vv{a} \times \vv{b}) \perp \vv{a} \land (\vv{a} \times \vv{b}) \perp \vv{b}$

\subsubsection{Satz}
Der Flächeninhalt des Parallelogramms, das von den Vektoren $\vv{a}, \ \vv{b} \neq \vv{0}$ unter dem Winkel $\varphi$ aufgespannt wird, beträgt $\mleft| \vv{a} \times \vv{b} \mright| = |\vv{a}| \cdot |\vv{b}| \cdot \sin(\varphi)$. \par
Das Volumen des Spates, der von den Vektoren $\vv{a}, \ \vv{b}, \ \vv{c} \neq \vv{0}$ aufgespannt wird, beträgt $\mleft| \mleft( \vv{a} \times \vv{b}\mright) \circ \vv{c} \mright|$.







\end{document}